\section{Open-loop planning via H-steps ahead predictions}
\label{sec:open_loop_planning}
The proposed approach requires the presence of $H+1$ versions of the matrix $\bP$ at each grid point of the discretized problem. One of these versions is initialized with $\bar{\bP}_0$, while the others represent the evolution of matrices that were initialized at previous grid points, from 1 to $H$ steps earlier. This ensures that at each step it is possible to find a covariance matrix evolved for $H$ steps and use it to robustify the constraints. The initialization of the covariance matrix is necessary due to its divergent dynamics. In fact, without a feedback control, the error on the position and orientation of the body reference frame can only increase along the track, compromising the conditioning of the matrix. 

%\subsection{Discretization via direct collocation}
%\label{sec:openloopdiscretization}
Introducing the versioning of the covariance matrix approach requires a modification of the Equations described in Section~\ref{sec:discretization}. In particular Eqs.~\eqref{eq:DOCPdP}, \eqref{eq:DOCPdPIC}, and \eqref{eq:DOCPconstraints} are replaced by Eqs.~\eqref{eq:DOCPdPopenloop}, \eqref{eq:DOCPdPinitopenloop}, and \eqref{eq:DOCPconstraintsopenloop}. Eq.~\eqref{eq:DOCPdPopenloop} collects the continuity and collocation Equations for all the versions of the covariance matrix that have propagated from the previous step. To account for the evolution of $H$ matrices, it is necessary to introduce $H\times d$ matrices $\bSi_k^j=(\bSi_{k,1}^j, \ldots, \bSi_{k,d}^j)$, which represent the values of each matrix version at the collocation points. 

The specific formulation is expressed as follows:
\begin{alignat}{3}
	\underset{\bmu_k,\bxi_k, \bu_k, \bP_k,\bz_k}{\text{minimize}} \,
	& & & J_k(\bmu_k,\bxi_k, \bu_k) & & \label{eq:DOCPcost} \\
	\hspace*{-2.0 cm}\text{s.t.} \quad
	& \bzero      & = & \; \bPsimu_k(\bmu_{k-1},\bmu_k,\bxi_k, \bu_k,\bz_k),
	& \quad & k = 1,\ldots, N \label{eq:DOCPdyn} \\
	& \bmu_0      & = & \; \bar{\bmu}_0
	& & \label{eq:DOCPdynIC} \\
	& \bzero      & = & \; \bPsiP_k(\bmu_k,\bxi_k, \bu_k, \bP_{k-1}^{j-1},\bP_k^j,\bSi_k^j,\bz_k),
	& \quad & \substack{k = 1,\ldots, N; \; \\ j=1,\ldots,H;\;} \label{eq:DOCPdPopenloop} \\
	& \bP_k^0       & = & \; \bar{\bP}_0 \succeq 0
	& \quad & k = 1,\ldots, N; \; \label{eq:DOCPdPinitopenloop} \\
	& \bzero      & = & \; \bOm_k(\bmu_k,\bxi_k, \bu_k,\bz_k),
	& \quad & k = 0,\ldots, N \label{eq:DOCPpath} \\
	& 0           & \geq & \; h_i(\bmu_k, \bu_k, \bz_k) + \be_i(\bmu_k, \bu_k, \bP_k^H, \bz_k),
	& \quad & k = 1,\ldots, N;\; i \in \calI \label{eq:DOCPconstraintsopenloop}
\end{alignat}

Eq.~\eqref{eq:DOCPdPinitopenloop} is used to initialize the correct version at the $k$-th step while Eq.~\eqref{eq:DOCPconstraintsopenloop} specify that the back-off term of the constraints is evaluated with the most propagated covariance matrix version, $\bP^H_k$.
A schematic representation of the approach used to manage the $H+1$ versions of the covariance matrix is shown in Figure \ref{fig:DOCPgrid}, where the dashed rectangle represent the $k$-th step at which the constraints are being formulated. At each grid point, two versions of the covariance matrix are highlighted: $\bP^0_k$ (red node), which is the version to be initialized, and $\bP^H_k$ (green node), which is the most propagated version used to formulate the robust constraints in Eq.~\eqref{eq:DOCPconstraintsopenloop}.

\begin{figure}
	\centering
	\begin{tikzpicture}[%
		smallnode/.style={%
			circle, draw, minimum size=3mm, inner sep=0pt
		},
		r_smallnode/.style={%
			circle, draw, minimum size=3mm, inner sep=0pt, red
		},
		g_smallnode/.style={%
			circle, draw, minimum size=3mm, inner sep=0pt, mygreen
		},
		every path/.style={->, thick},
		x=1.5cm, y=1.5cm
		]
		
		\newcommand{\Plab}[2]{\bP^{#2}_{#1}}
		\definecolor{mygreen}{RGB}{0 150 0}
		
		% Nodes
		% First row
		\node[smallnode, label={[label distance=-2pt]above:\protect\(\Plab{k-2}{H-2}\)}] (Pk-2H-2) at (1,0) {};
		\node[smallnode, label={[label distance=-2pt]above:\(\Plab{k-1}{H-1}\)}] (Pk-1H-1) at (2,0) {};
		\node[g_smallnode, label={[label distance=-2pt]above:\(\Plab{k}{H}\)}] (PkH) at (3,0) {};
		\node[r_smallnode, label={[label distance=-2pt]above:\(\Plab{k+1}{0}\)}] (Pk+10) at (4,0) {};
		\node[smallnode, label={[label distance=-2pt]above:\(\Plab{k+2}{1}\)}] (Pk+21) at (5,0) {};
		
		% Second row
		\node[smallnode, label={[label distance=-2pt]above:\(\Plab{k-2}{H-3}\)}] (Pk-2H-3) at (1,1) {};
		\node[smallnode, label={[label distance=-2pt]above:\(\Plab{k-1}{H-2}\)}] (Pk-1H-2) at (2,1) {};
		\node[smallnode, label={[label distance=-2pt]above:\(\Plab{k}{H-1}\)}] (PkH-1) at (3,1) {};
		\node[g_smallnode, label={[label distance=-2pt]above:\(\Plab{k+1}{H}\)}] (Pk+1H) at (4,1) {};
		\node[r_smallnode, label={[label distance=-2pt]above:\(\Plab{k+2}{0}\)}] (Pk+20) at (5,1) {};
		
		% dots row 
		\node[label={center,rotate=90:\(\dots\)}] (dots2) at (1, 1.7) {};
		\node[label={center,rotate=90:\(\dots\)}] (dots3) at (2, 1.7) {};
		\node[label={center,rotate=90:\(\dots\)}] (dots4) at (3, 1.7) {};
		\node[label={center,rotate=90:\(\dots\)}] (dots5) at (4, 1.7) {};
		\node[label={center,rotate=90:\(\dots\)}] (dots6) at (5, 1.7) {};
		
		% Third row 
		\node[g_smallnode, label={[label distance=-2pt]above:\(\Plab{k-2}{H}\)}] (Pk-2H) at (1,2) {};
		\node[r_smallnode, label={[label distance=-2pt]above:\(\Plab{k-1}{0}\)}] (Pk-10) at (2,2) {};
		\node[smallnode, label={[label distance=-2pt]above:\(\Plab{k}{1}\)}] (Pk1) at (3,2) {};
		\node[smallnode, label={[label distance=-2pt]above:\(\Plab{k+1}{2}\)}] (Pk+12) at (4,2) {};
		\node[smallnode, label={[label distance=-2pt]above:\(\Plab{k+2}{3}\)}] (Pk+23) at (5,2) {};
		
		% Fourth row
		\node[smallnode, label={[label distance=-2pt]above:\(\Plab{k-2}{H-1}\)}] (Pk-2H-1) at (1,3) {};
		\node[g_smallnode, label={[label distance=-2pt]above:\(\Plab{k-1}{H}\)}] (Pk-1H) at (2,3) {};
		\node[r_smallnode, label={[label distance=-2pt]above:\(\Plab{k}{0}\)}] (Pk0) at (3,3) {};
		\node[smallnode, label={[label distance=-2pt]above:\(\Plab{k+1}{1}\)}] (Pk+11) at (4,3) {};
		\node[smallnode, label={[label distance=-2pt]above:\(\Plab{k+2}{2}\)}] (Pk+22) at (5,3) {};
		
		% Curved arrows
		\draw[bend left=20] (Pk-2H-2) to (Pk-1H-1);
		\draw[bend left=20] (Pk-1H-1) to (PkH);
		\draw[bend left=20] (Pk+10) to (Pk+21);
		
		\draw[bend left=20] (Pk-2H-3) to (Pk-1H-2);
		\draw[bend left=20] (Pk-1H-2) to (PkH-1);
		\draw[bend left=20] (PkH-1) to (Pk+1H);
		
		\draw[bend left=20] (Pk-10) to (Pk1);
		\draw[bend left=20] (Pk1) to (Pk+12);
		\draw[bend left=20] (Pk+12) to (Pk+23);
		
		\draw[bend left=20] (Pk-10) to (Pk1);
		\draw[bend left=20] (Pk1) to (Pk+12);
		\draw[bend left=20] (Pk+12) to (Pk+23);
		
		\draw[bend left=20] (Pk-2H-1) to (Pk-1H);
		\draw[bend left=20] (Pk0) to (Pk+11);
		\draw[bend left=20] (Pk+11) to (Pk+22);
		
		% Dashed rectangle
		\draw[dashed, thick, rounded corners] (2.6,-.25) rectangle (3.4,3.5);
		
	\end{tikzpicture}
	\caption{Schematic representation of the continuity and initialization Equations for the $H+1$ versions of the covariance matrix introduced in the discretized OCP. The dashed rectangle indicate the $k$-th step, in which the constraints are formulated. Each node represents a covariance matrix, labeled such that the subscript denotes the current grid step, while the superscript indicates the number of propagation steps it has undergone. At each step, two nodes are highlighted: the covariance matrix to be initialized (red node) and the matrix that has been propagated for $H$ steps (green node), which is used to formulate the robust constraints.}
	\label{fig:DOCPgrid}
\end{figure}

\subsection{Robust adherence constraint}
