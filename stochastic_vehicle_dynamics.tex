\section{Stochastic vehicle dynamics}
\label{sec:stochastic_vehicle_dynamics}

We model the \emph{perturbed vehicle dynamics} as the following nonlinear continuous-time random dynamical system
\begin{align}
\dbx(t) = f(\bx(t), \bu(t)) + \bw(t),
\end{align}
where $\bu(t)$ are the deterministic control inputs, $\bw(t)$ is additive Gaussian white noise with zero mean and known covariance $\bQ(t)$, i.e. $\bw(t) \sim \calN(\bzero, \bQ(t))$, and the state $\bx(t)$ results in a Gaussian distribution with mean $\bmu(t)$ and covariance $\bP(t)$, so that $\bx(t) \sim \calN(\bmu(t), \bP(t))$.

Due to the symmetry of the probability density function (pdf) with respect to $\bmu(t)$, it is possible to represent the time evolution of the pdf as: i) the deterministic evolution of the mean $\bmu(t)$
\begin{align}\label{eq:meandynamics}
\dbmu(t) = f(\bmu(t), \bu(t)),
\end{align}
and ii) the time evolution of the state covariance matrix $\bP(t)$ along $\bmu(t)$, which can be expressed by the \emph{Lyapunov matrix differential equation}
\begin{align}\label{eq:dP}
\dbP(t) = \bA(t) \bP(t) + \bP(t) \bA^T(t) + \bQ(t),\quad \bP(0) = \bP_0 = \bP_0^T.
\end{align}
In~\eqref{eq:dP}, $\bP_0$ is the \emph{initial} state covariance and $\bA(t)$ is the usual shorthand notation for the Jacobian along the mean trajectory $\bmu(t)$, that is $\bA(t)=\bJ(\bmu(t),\bu(t))$, where $\bJ(\bx, \bu) = \frac{\pd f(\bx, \bu)}{\pd \bx}$.

As can be readily verified by differentiation~\cite{Gajic:LyapunovMatrixEquation:2010}, the analytical solution of~\eqref{eq:dP} has the form
\begin{align}\label{eq:P_STM}
\bP(t) = \bPhi(t,t_0) \bP_0 \bPhi(t,t_0)+\int_{t_0}^{t} \bPhi(t,\tau) \bQ(\tau) \bPhi^T(t,\tau) \dd \tau \quad \bP(0)=\bP_0,
\end{align}
where $\bPhi(t,t_0)$ is the \emph{state transition matrix}. This matrix encodes the evolution from $t_0$ to $t$ of a perturbation w.r.t. to a fiducial trajectory -- the mean $\bmu(t)$. In symbols, $\barbx(t) = \bPhi(t,t_0) \barbx(t_0)$, with $\barbx(\cdot) =\bx(\cdot) - \bmu(\cdot) $. In turn, the evolution of $\bPhi(t,t_k)$ from a generic $t_k$ to $t$ is driven by the following differential equation
\begin{align}\label{eq:STM}
\dbPhi(t,t_k) = \bA(t)\bPhi(t,t_k), \quad \bPhi(t_k, t_k) = \bI.
\end{align}
In geometric and orbital mechanics, see e.g.~\cite{Maruskin:DynamicalSystemsGeometric:2018} or~\cite{Tapley:StatisticalOrbitDetermination:2004}, the usual choice for statistical trajectory determination is to employ~\eqref{eq:meandynamics} along with~\eqref{eq:P_STM} and~\eqref{eq:STM}. In our case, since we use collocation integrators for stochastic trajectory planning, we follow a more direct approach -- motivated by~\cite{Gillis:PracticalMethodsApproximate:2015} -- by directly employing~\eqref{eq:meandynamics} and~\eqref{eq:dP}.

Accordingly, the continuous-time stochastic trajectory planning can be framed as the following nonlinear optimal control problem
\begin{align} 
& \underset{\bmu(t), \bu(t), \bP(t)}{\text{minimize}} \quad J(\bmu(t), \bu(t), \bP(t)) \label{eq:OCPcost} \\
& \text{s.t.} \quad \dbmu(t)           = f(\bmu(t), \bu(t)) \label{eq:OCPdyn} \\
& \phantom{\text{s.t.} \quad} \bmu(0)  = \bmu_0 \label{eq:OCPdynIC} \\
& \phantom{\text{s.t.} \quad} \dbP(t) = \bA(t) \bP(t) + \bP(t) \bA^T(t) + \bQ(t) \label{eq:OCPdP} \\
& \phantom{\text{s.t.} \quad} \bP(0)  = \bP_0 \succeq 0 \label{eq:OCPdPIC} \\ %& \phantom{\text{s.t.} \qquad} 0       \geq h_i(\bmu(t), \bu(t))
%+ \overbrace{\gamma \bigg[\underbrace{\na^T_{\bx} h_i \bP(t) \na_{\bx} h_i}_{\text{variance of $h_i(\bx)$}}\bigg]^{\frac{1}{2}}}^{\text{safety margin}},
&\phantom{\text{s.t.} \qquad} 0       \geq h_i(\bmu(t), \bu(t))
+ \be_i(\bmu(t), \bu(t), \bP(t)),
\quad i \in \calI \label{eq:OCPconstraints}
\end{align}
The cost function $J$ in~\eqref{eq:OCPcost} depends on the mean $\bmu(t)$, the controls $\bu(t)$ and the state covariance $\bP(t)$. The mean dynamics is expressed by~\eqref{eq:OCPdyn} with~\eqref{eq:OCPdynIC}, and the covariance dynamics is expressed by~\eqref{eq:OCPdP} with~\eqref{eq:OCPdPIC}. In eq.~\eqref{eq:OCPconstraints} the \emph{backoff terms} $\be_i$ account for the disturbances and serve the purpose of obtaining deterministic safety-margins directly on $\bmu(t)$. $\calI$ is the set of indices defining the inequality constraints. This is equivalent to a linearization around the mean of the original \emph{chance constraint} on $\bx(t)$ expressed by $\Prob \{h_i(\bx) \leq 0\}\geq p$, where $p$ is the confidence level of constraint satisfaction. The backoff terms are given by
$\be_i = \gamma \sig_i$. The coefficient $\gamma = \Phi^{-1}(p)$ is the quantile function, where $\Phi(z)=\Prob \{Z\leq z\}$ is the cdf of a standard normal distribution $Z \sim \calN(0,1)$, and acts as a \emph{tuning knob}: the greater the confidence level $p$ required, the higher the gain $\ga$\footnote{For example, with $p=0.84$ $\ga = 1.0$, with $p=0.97$ $\ga = 2.0$, with $p=0.99$ $\ga = 3.0$}. The term $\sig_i = \big[\na^T_{\bx} h_i(\bmu) \bP(t) \na_{\bx} h_i(\bmu)\big]^{\frac{1}{2}}$ represents the standard deviation of the constraint $h_i(\bx)$ linearized around the mean, i.e. of random variable $h_i(\bmu)+\na^T_{\bx} h_i(\bmu) (\bx - \bmu)$, and follows immediately from the propagation rule of covariance.

\subsection{Discretization via direct collocation}
\label{sec:discretization}

The nonlinear optimal control problem~\eqref{eq:OCPcost} can be discretized by applying a suitable collocation integrator obtaining the following nonlinear program (NLP) 
\begin{alignat}{3}
\underset{\bmu_k,\bxi_k, \bu_k, \bP_k,\bz_k}{\text{minimize}} \,
& & & J_k(\bmu_k,\bxi_k, \bu_k) & & \label{eq:DOCPcost} \\
\hspace*{-2.0 cm}\text{s.t.} \quad
& \bzero      & = & \; \bPsimu_k(\bmu_{k-1},\bmu_k,\bxi_k, \bu_k,\bz_k),
& \quad & k = 1,\ldots, N \label{eq:DOCPdyn} \\
& \bmu_0      & = & \; \bar{\bmu}_0
& & \label{eq:DOCPdynIC} \\
& \bzero      & = & \; \bPsiP_k(\bmu_k,\bxi_k, \bu_k, \bP_{k-1},\bP_k,\bSi_k,\bz_k),
& \quad & k = 1,\ldots, N \label{eq:DOCPdP} \\
& \bP_0       & = & \; \bar{\bP}_0 \succeq 0
& & \label{eq:DOCPdPIC} \\
& \bzero      & = & \; \bOm_k(\bmu_k,\bxi_k, \bu_k,\bz_k),
& \quad & k = 0,\ldots, N \label{eq:DOCPpath} \\
& 0           & \geq & \; h_i(\bmu_k, \bu_k, \bz_k) + \be_i(\bmu_k, \bu_k, \bP_k, \bz_k),
& \quad & k = 1,\ldots, N;\; i \in \calI \label{eq:DOCPconstraints}
\end{alignat}
To perform the discretization, the track centerline is parameterized using a curvilinear abscissa $\alpha \in [0,1]$, and uniformly sampled at $N+1$ points $\alpha_0, \ldots, \alpha_N$.
Accordingly, $\bmu_k$ denotes the mean state at grid node $\alpha_k$, while $\bu_k$ and $\bz_k$ are assumed to be piecewise constant over each interval $[\al_k, \al_{k+1}]$.
These represent the controls and the algebraic variables, respectively. The $\bz_k$'s are introduced as direct handles for physically meaningful quantities such contact forces. Similarly, $\bP_k$ represents the covariance matrix at the $k$-th node.
Following the direct collocation approach, both the mean and covariance state trajectories are approximated, in the $k$-th interval, with polynomials $\pi_k(\tau)$, defined on the unit interval $\tau\in[0,1]$, and then scaled to match the width $h_k$ of the corresponding time step. On the unit interval, we select $d$ collocation points $\tau_1, \ldots, \tau_d$, associated with as many collocation states. Accordingly, we define $\bxi_k$ and $\bSi_k$ as the mean states and covariance matrices at the $d$ collocation points within each interval $[\al_k, \al_{k+1}]$, respectively. Therefore, $\bxi_k =
(\bxi_{k,1}, \ldots, \bxi_{k,d})$, and similarly $\bSi_k=(\bSi_{k,1}, \ldots, \bSi_{k,d})$. Equations~\eqref{eq:DOCPdyn} and~\eqref{eq:DOCPdP} represent the collocation and continuity equations for mean and covariance, respectively, with initial conditions represented by~\eqref{eq:DOCPdynIC} and~\eqref{eq:DOCPdPIC}. Eqs.~\eqref{eq:DOCPpath} are path equality constraints and~\eqref{eq:DOCPconstraints} are the robustified inequality constraints. 

In certain cases~\cite{Gillis:PracticalMethodsApproximate:2015}, positive-definiteness-preserving Lyapunov discretization schemes are used in~\eqref{eq:DOCPdP}. However, for sufficiently fine discretization, we found that integrating only the lower triangular part of $\bP(t)$ in~\eqref{eq:OCPdPIC}  (and reconstructing its strictly upper part accordingly), ensures that all $\bP_k$ remain symmetric and positive definite when the initial $\bP_0$ is symmetric and positive definite. In all our tests direct collocation with cubic polynomial state representations and Gauss-Legendre collocation points were used.
