\section{Conclusions} 
\label{sec:conclusions}

This work has proposed and compared two robust planning strategies for minimum-time trajectory optimization in motorsport scenarios under uncertainty. The first approach, based on open-loop horizon-based covariance propagation, relies on a worst-case analysis of disturbance growth over a finite horizon to conservatively back off safety constraints. The second, a closed-loop covariance-aware method, incorporates a time-varying LQR feedback policy into the planning process, yielding a more realistic estimate of uncertainty evolution and tighter constraint enforcement.

Quantitative results on a representative circuit segment have highlighted the trade-offs between robustness and performance. Open-loop planning, while conservative, ensures feasibility even in the absence of feedback, but incurs a higher performance cost due to larger safety margins. The closed-loop method achieves comparable safety guarantees with reduced conservativeness and improved lap-time performance, thanks to its ability to explicitly model the stabilizing action of the driver. 
Finally, random simulations with noise realization performed adopting the closed-loop approach proved the effectiveness of the robustified constraints. A LQR-based controller tracking a robust trajectory planned with probabilistic constraint tightening exhibits significantly fewer track violations compared to one tracking the nominal optimal trajectory.

Overall, the proposed framework enables the design of minimum-time trajectories that are not only fast but also certified to be safe under realistic disturbances. By embedding robustness into the planning stage, the gap between theoretical optimality and real-world executability is narrowed. Future work will explore the generalization to more accurate vehicle models and experimental validation of the methodology.