\section{TMP NAME}
\label{sec:tmp}

\subsection{Vehicle model}
\label{sec:vehicle_model}
The vehicle model employed in this study is a Single Track model featuring nonlinear axle characteristics. The lateral forces $Y_1$ and $Y_2$ are computed using a Pacejka's Magic Formula, which depends on the total vertical load acting on the axle and the theoretical lateral slip evaluated at the midpoint of the axle. 
The longitudinal forces $X_1$ and $X_2$ are provided as inputs to the system. Specifically, the first two components of the input vector $\bu$ correspond to acceleration and braking forces, while the third component represents the steering angle of the front wheels. This input formulation facilitates the distribution of braking forces between the front and rear axles, while assigning acceleration exclusively to the rear axle, consistent with the rear-wheel-drive configuration of the modeled vehicle. 

By treating the longitudinal forces as inputs, the longitudinal dynamics can be explicitly solved, allowing the computation of vertical load transfers prior to their use in the expression of the lateral forces. Consequently, the time derivative of the state becomes an explicit function of the state vector $\bx$ and the input vector $\bu$, as follows:
\begin{equation}
	\dbx(t) = f(\bx(t), \bu(t))
\end{equation}

The state vector $\bx$ has six components and includes the three kinematic quantities $u$, $v$, and $r$, along with the position and orientation of the barycentrical reference frame. As illustrated in Figure~\ref{fig:vehicle_model} $u$ and $v$ represent the longitudinal and lateral velocity of the CoM w.r.t its reference frame, while $r$ represents vehicle's yaw rate.

\begin{figure}
	\centering
	\begin{tikzpicture}[scale=2, x = {(0,1cm)}, y = {(-1cm,0)},
		force/.style={->, thick, red!60!black, >={Latex[length=10pt, width=4pt]}},
		vec/.style={->, thick, mgreen, >={Latex[length=10pt, width=4pt]}},
		aux/.style={thin},
		frame/.style={->,thin,>={Latex}},
		every node/.style={font=\small, text=black}
		]
		\definecolor{mgreen}{RGB}{0,100,0}
		\begin{scope}[rotate = 25,shift = {(-.3,-2)}]
			% Coordinates
			\coordinate (rear) at (0,0);
			\coordinate (G) at (1.2,0);
			\coordinate (front) at (2.8,0);
			
			% Main body line
			\draw[blue!50!black, thick] (rear) -- (front);
			
			% Rear wheel
			\draw[fill=black!60!white, draw=black, rounded corners=2pt]
			($(rear)+(-.2,.1)$) rectangle ($(rear)+(+.2,-.1)$);
			
			% Front wheel (rotated around its center)
			\begin{scope}[shift={(front)}, rotate=25]
				\draw[fill=black!60!white, draw=black, rounded corners=2pt]
				(-0.2,.1) rectangle (0.2,-.1);
			\end{scope}
			
			% Local axes at rear
			\draw[force] (rear) -- ++(.5,0) node[above,right] {$X_2$};
			\draw[force] (rear) -- ++(0,1.3) node[above] {$Y_2$};
			
			% Local axes at front
			\draw[force] (front) -- ++(-.3,0) node[right] {$X_1$};
			\draw[force] (front) -- ++(0,1.5) node[above] {$Y_1$};
			
			% Velocity vectors at G
			\draw[vec] (G) -- ++(.6,0) node[left, shift={(-.2,-.1)}] {$u\,\mathbf{i}$};
			\draw[vec] (G) -- ++(0,.3) node[above] {$v\,\mathbf{j}$};
			
			% Angular velocity
			\begin{scope}[rotate = -115]
				\draw[vec] 
				(-.1,.8) arc[start angle=0, end angle=50, radius=1]
				node[midway, right=2pt] {$r\,\mathbf{k}$};
			\end{scope}
			
			% G point
			\fill (G) circle (1pt) node[right] {$G$};
			
			% Body frame
			\draw[frame] (G) --++ (1,0) node[right] {$x_b$};
			\draw[frame] (G) --++ (0,1) node[above] {$y_b$};
			
			% Distances
%			\draw[<->,>={Latex}] (3,-1) -- node[left] {$a_1$} (1.5,-1);
%			\draw[<->,>={Latex}] (1.5,-1) -- node[left] {$a_2$} (0,-1);
			
			% Steering angle (auxiliary vertical line)
			\draw[aux] (front) -- ++(.4,0);
			\draw[aux] (front) -- ++({0.4*cos(25)}, {0.4*sin(25)});
			
			% Steering angle arc
			\draw[->,thin,>={Latex}] ([shift={(0.35,0)}]front) arc[start angle=0, end angle=25, radius=0.35];
			\node[left,shift={(.15,0)}] at ([shift={(0.35,0)}]front) {$\delta$};
			
			% Psi angle aux
			\draw[aux] (G) -- ++({0.4*cos(25)}, {-0.4*sin(25)});
			
			% Psi angle
			\draw[->,thin,>={Latex}] ([shift={({0.35*cos(25)},{-.35*sin(25)})}]G) arc[start angle=-25, end angle=0, radius=0.35];
			\node[left,shift={(.2,-.05)}] at ([shift={({0.35*cos(25)},{-.35*sin(25)})}]G) {$\psi$};
			
		\end{scope}
		\coordinate (ground) at (0,0);
		\draw[frame] (ground) --++ (1,0) node[right] {$x$};
		\draw[frame] (ground) --++ (0,1) node[above] {$y$};
		\draw[->,thick,>=Latex] (ground) --++ (G);
%		\draw[->,thick,>=Latex] (ground) --++ (G) node[midway, shift={(0,-.3)}] {$\boldsymbol{p}$};
		
	\end{tikzpicture}
	\caption{Single Track model. The ground force acting on each axle is decomposed into longitudinal and lateral components, indicated by red arrows. The longitudinal components are part of the input vector $\bu$, encompassing both acceleration and braking forces, while the lateral components are computed using Pacejka's Magic Formula based on the vertical load and slip angle at each axle. The figure also depicts the fixed reference frame and the body reference frame, with the origin located at the vehicle's center of mass (CoM). The state variables $u$ and $v$ represent the longitudinal and lateral velocities of the CoM w.r.t. the body frame, while $r$ denotes the yaw rate. The angle $\psi$ is used to describe the orientation of the body frame w.r.t. the fixed frame.}
	\label{fig:vehicle_model}
\end{figure}

\subsection{Stochastic vehicle dynamics}
\label{sec:stochastic_vehicle_dynamics}

We model the \emph{perturbed vehicle dynamics} as the following nonlinear continuous-time random dynamical system
\begin{align}
\dbx(t) = f(\bx(t), \bu(t)) + \bw(t),
\end{align}
where $\bu(t)$ are the deterministic control inputs, $\bw(t)$ is additive Gaussian white noise with zero mean and known covariance $\bQ(t)$, i.e. $\bw(t) \sim \calN(\bzero, \bQ(t))$, and the state $\bx(t)$ results in a Gaussian distribution with mean $\bmu(t)$ and covariance $\bP(t)$, so that $\bx(t) \sim \calN(\bmu(t), \bP(t))$.

Due to the symmetry of the probability density function (pdf) with respect to $\bmu(t)$, it is possible to represent the time evolution of the pdf as: i) the deterministic evolution of the mean $\bmu(t)$
\begin{align}\label{eq:meandynamics}
\dbmu(t) = f(\bmu(t), \bu(t)),
\end{align}
and ii) the time evolution of the state covariance matrix $\bP(t)$ along $\bmu(t)$, which can be expressed by the \emph{Lyapunov matrix differential equation}
\begin{align}\label{eq:dP}
\dbP(t) = \bA(t) \bP(t) + \bP(t) \bA^T(t) + \bQ(t),\quad \bP(0) = \bP_0 = \bP_0^T.
\end{align}
In~\eqref{eq:dP}, $\bP_0$ is the \emph{initial} state covariance and $\bA(t)$ is the usual shorthand notation for the Jacobian along the mean trajectory $\bmu(t)$, that is $\bA(t)=\bJ(\bmu(t),\bu(t))$, where $\bJ(\bx, \bu) = \frac{\pd f(\bx, \bu)}{\pd \bx}$.

As can be readily verified by differentiation~\cite{Gajic:LyapunovMatrixEquation:2010}, the analytical solution of~\eqref{eq:dP} has the form
\begin{align}\label{eq:P_STM}
\bP(t) = \bPhi(t,t_0) \bP_0 \bPhi(t,t_0)+\int_{t_0}^{t} \bPhi(t,\tau) \bQ(\tau) \bPhi^T(t,\tau) \dd \tau \quad \bP(0)=\bP_0,
\end{align}
where $\bPhi(t,t_0)$ is the \emph{state transition matrix}. This matrix encodes the evolution from $t_0$ to $t$ of a perturbation w.r.t. to a fiducial trajectory -- the mean $\bmu(t)$. In symbols, $\barbx(t) = \bPhi(t,t_0) \barbx(t_0)$, with $\barbx(\cdot) =\bx(\cdot) - \bmu(\cdot) $. In turn, the evolution of $\bPhi(t,t_k)$ from a generic $t_k$ to $t$ is driven by the following differential equation
\begin{align}\label{eq:STM}
\dbPhi(t,t_k) = \bA(t)\bPhi(t,t_k), \quad \bPhi(t_k, t_k) = \bI.
\end{align}
In geometric and orbital mechanics, see e.g.~\cite{Maruskin:DynamicalSystemsGeometric:2018} or~\cite{Tapley:StatisticalOrbitDetermination:2004}, the usual choice for statistical trajectory determination is to employ~\eqref{eq:meandynamics} along with~\eqref{eq:P_STM} and~\eqref{eq:STM}. In our case, since we use collocation integrators for stochastic trajectory planning, we follow a more direct approach -- motivated by~\cite{Gillis:PracticalMethodsApproximate:2015} -- by directly employing~\eqref{eq:meandynamics} and~\eqref{eq:dP}.

Accordingly, the continuous-time stochastic trajectory planning can be framed as the following nonlinear optimal control problem
\begin{subequations}\label{eq:OCP}
\begin{align}
& \underset{\bmu(t), \bu(t), \bP(t)}{\text{minimize}} \quad J(\bmu(t), \bu(t), \bP(t)) \label{eq:OCPcost} \\
& \text{s.t.} \quad \dbmu(t)           = f(\bmu(t), \bu(t)) \label{eq:OCPdyn} \\
& \phantom{\text{s.t.} \quad} \bmu(0)  = \bmu_0 \label{eq:OCPdynIC} \\
& \phantom{\text{s.t.} \quad} \dbP(t) = \bA(t) \bP(t) + \bP(t) \bA^T(t) + \bQ(t) \label{eq:OCPdP} \\
& \phantom{\text{s.t.} \quad} \bP(0)  = \bP_0 \succeq 0 \label{eq:OCPdPIC} \\ %& \phantom{\text{s.t.} \qquad} 0       \geq h_i(\bmu(t), \bu(t))
%+ \overbrace{\gamma \bigg[\underbrace{\na^T_{\bx} h_i \bP(t) \na_{\bx} h_i}_{\text{variance of $h_i(\bx)$}}\bigg]^{\frac{1}{2}}}^{\text{safety margin}},
&\phantom{\text{s.t.} \qquad} 0       \geq h_i(\bmu(t), \bu(t))
+ \be_i(\bmu(t), \bu(t), \bP(t)),
\quad i \in \calI \label{eq:OCPconstraints}
\end{align}
\end{subequations}
The cost function $J$ in~\eqref{eq:OCPcost} depends on the mean $\bmu(t)$, the controls $\bu(t)$ and the state covariance $\bP(t)$. The mean dynamics is expressed by~\eqref{eq:OCPdyn} with~\eqref{eq:OCPdynIC}, and the covariance dynamics is expressed by~\eqref{eq:OCPdP} with~\eqref{eq:OCPdPIC}. In eq.~\eqref{eq:OCPconstraints} the \emph{backoff terms} $\be_i$ account for the disturbances and serve the purpose of obtaining deterministic safety margins directly on $\bmu(t)$. $\calI$ is the set of indices defining the inequality constraints. The backoff terms stem for a linearization around the mean of the original \emph{chance constraint} on $\bx(t)$ expressed by $\Prob \{h_i(\bx) \leq 0\}\geq p$, where $p$ is the confidence level of constraint satisfaction. Explicitly the backoff terms are given by
$\be_i = \gamma \sig_i$. The coefficient $\gamma = \Phi^{-1}(p)$ is the quantile function, where $\Phi(z)=\Prob \{Z\leq z\}$ is the cdf of a standard normal distribution $Z \sim \calN(0,1)$, and acts as a \emph{tuning knob}: the greater the confidence level $p$ required, the higher the gain $\ga$\footnote{For example, with $p=0.84$ $\ga = 1.0$, with $p=0.97$ $\ga = 2.0$, with $p=0.99$ $\ga = 3.0$}. The term $\sig_i = \big[\na^T_{\bx} h_i(\bmu) \bP(t) \na_{\bx} h_i(\bmu)\big]^{\frac{1}{2}}$ represents the standard deviation of the constraint $h_i(\bx)$ linearized around the mean, i.e. of random variable $h_i(\bmu)+\na^T_{\bx} h_i(\bmu) (\bx - \bmu)$, and follows immediately from the propagation rule of covariance.

\subsection{Discretization via direct collocation}
\label{sec:discretization}

The nonlinear optimal control problem~\eqref{eq:OCP} can be discretized by applying a suitable collocation integrator obtaining the following nonlinear program (NLP)
\begin{subequations}\label{eq:DOCP}
\begin{alignat}{3}
\underset{\bmu_k,\bxi_k, \bu_k, \bP_k,\bz_k}{\text{minimize}} \,
& & & J_k(\bmu_k,\bxi_k, \bu_k) & & \label{eq:DOCPcost} \\
\hspace*{-2.0 cm}\text{s.t.} \quad
& \bzero      & = & \; \bPsimu_k(\bmu_{k-1},\bmu_k,\bxi_k, \bu_k,\bz_k),
& \quad & k = 1,\ldots, N \label{eq:DOCPdyn} \\
& \bmu_0      & = & \; \bar{\bmu}_0
& & \label{eq:DOCPdynIC} \\
& \bzero      & = & \; \bPsiP_k(\bmu_k,\bxi_k, \bu_k, \bP_{k-1},\bP_k,\bSi_k,\bz_k),
& \quad & k = 1,\ldots, N \label{eq:DOCPdP} \\
& \bP_0       & = & \; \bar{\bP}_0 \succeq 0
& & \label{eq:DOCPdPIC} \\
& \bzero      & = & \; \bOm_k(\bmu_k,\bxi_k, \bu_k,\bz_k),
& \quad & k = 0,\ldots, N \label{eq:DOCPpath} \\
& 0           & \geq & \; h_i(\bmu_k, \bu_k, \bz_k) + \be_i(\bmu_k, \bu_k, \bP_k, \bz_k),
& \quad & k = 1,\ldots, N;\; i \in \calI \label{eq:DOCPconstraints}
\end{alignat}
\end{subequations}
To perform the discretization, the track centerline is parameterized using a curvilinear abscissa $\alpha \in [0,1]$, and uniformly sampled at $N+1$ points $\alpha_0, \ldots, \alpha_N$.
Accordingly, $\bmu_k$ denotes the mean state at grid node $\alpha_k$, while the controls $\bu_k$ and the algebraic variables $\bz_k$ are assumed to be piecewise constant over each interval $[\al_k, \al_{k+1}]$.
The $\bz_k$'s are introduced as direct handles for physically meaningful quantities such contact forces. Similarly, $\bP_k$ represents the covariance matrix at the $k$-th node.
Following the direct collocation approach, both the mean and covariance state trajectories are approximated, in the $k$-th interval, with polynomials $\pi_k(\tau)$, defined on the unit interval $\tau\in[0,1]$, and then scaled to match the width $h_k$ of the corresponding time step. On the unit interval, we select $d$ collocation points $\tau_1, \ldots, \tau_d$, associated with as many collocation states. Accordingly, we define $\bxi_k$ and $\bSi_k$ as the mean states and covariance matrices at the $d$ collocation points within each interval $[\al_k, \al_{k+1}]$, respectively. Therefore, $\bxi_k =
(\bxi_{k,1}, \ldots, \bxi_{k,d})$, and similarly $\bSi_k=(\bSi_{k,1}, \ldots, \bSi_{k,d})$. Equations~\eqref{eq:DOCPdyn} and~\eqref{eq:DOCPdP} represent the collocation and continuity equations for mean and covariance, respectively, with initial conditions represented by~\eqref{eq:DOCPdynIC} and~\eqref{eq:DOCPdPIC}. Eqs.~\eqref{eq:DOCPpath} are path equality constraints and~\eqref{eq:DOCPconstraints} are the robustified inequality constraints.

In certain cases~\cite{Gillis:PracticalMethodsApproximate:2015}, positive-definiteness-preserving Lyapunov discretization schemes are used in~\eqref{eq:DOCPdP}. However, for sufficiently fine discretization, we found that integrating only the lower triangular part of $\bP(t)$ in~\eqref{eq:OCPdPIC}  (and reconstructing its strictly upper part accordingly), ensures that $\bP_k$ remain symmetric and positive definite when the initial $\bP_0$ is symmetric and positive definite. In all our tests direct collocation with cubic polynomial state representations and Gauss-Legendre collocation points provided accurate results.
