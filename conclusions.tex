\section{Conclusions}
\label{sec:conclusions}

In this work we investigated the potential of some new technologies in empowering next generation autonomous race vehicles. In particular, the upgrades considered here were: four independent steering wheels (All Steering - AS), four independent wheel torques (All Independent Wheel Drive - AIWD), and independent control of the four camber angles (Dynamic Camber Control - DCC).

Firstly, we discussed the impact of enabling these control inputs on the vehicle performance, in relation to the mass increase associated to the new components needed for the systems.
The results confirmed that the vehicle endowed with all the features (All Independent Wheel Drive, All Steering, Dynamic Camber Control - AIWD AS DCC) is the most performant and forgiving to added mass. Compared to a traditional vehicle (Rear Wheel Drive, Forward Steering - RWD FS), its advantage in lap time persists until 123.5\,kg of added mass.

Then, lap times are presented for the baseline (Rear Wheel Drive, Forward Steering - RWD FS), and for vehicles with single upgrades added, keeping constant the total mass.
In particular, the configuration analyzed are: Rear Wheel Drive, All Steering - RWD AS, All Independent Wheel Drive, Forward Steering - AIWD FS, and Rear Wheel Drive, Forward Steering, Dynamic Camber Control - RWD FS DCC.
From these results it is evident that the single feature that improves the performance the most is the AIWD FS in Siena circuit, and the RWD FS DCC in N\"{u}rburgring circuit.

Focusing on trajectories, we can highlight the following features. The All Independent Wheel Drive, Forward Steering (AIWD FS) vehicle, being able to perform torque vectoring, travels turns closer to the inner boundary of the track, compared to a traditional vehicle, and exits with an higher forward speed.

A Rear Wheel Drive, All Steering (RWD AS) vehicle experiences in-phase rear steering for high speed turns, and counter-phase rear steering for low speed turns, hence improving cornering in each condition.

Being able to control the camber dynamically, a Rear Wheel Drive Vehicle, Forward Steering, Dynamic Camber Control (RWD FS DCC) vehicle, positions each wheel w.r.t. the road to raise the force limits, thus maximizing the ground forces.

To highlight the impact of the torque vectoring and camber control, a vehicle with these characteristics but \emph{without steering abilities} (All Independent Wheel Drive, Zero Steering, Dynamic Camber Control - AIWD 0S DCC) has been analyzed.
This vehicle completes a lap in Siena 3.66 seconds faster than a traditional vehicle (RWD FS), and corners effectively by applying higher torques on outer wheels than the inner wheels.

Introducing a temperature-dependent tyre model, we observed that the least affected vehicle is the traditional RWD FS, which stresses the wheels the least and thus loses only 11 ms due to the drop in grip factor. In contrast, All Independent Wheel Drive (AIWD) vehicles tend to stress all four wheels much more. As a result, when the tire thermal model in the MLTP takes into account the reduction in grip with increasing temperature, AIWD vehicles are penalized more than RWD FS vehicles, losing approximately 100 ms each in the considered sector.

\section{Future Work}

While this study provides a broad overview of the possibilities and implications of introducing certain features to a new generation of autonomous vehicles, some aspects require further investigation.

A more complete tire model, which could account for tread wear and the consequent inevitable performance drop due to tire usage, should be implemented. To achieve this, either a testing campaign on the actual modeled tires to fine-tune the parameters of the wear-dependent phenomena, or a new dataset of fully characterized tires, is essential.

The introduction of realistic aerodynamic maps could indeed lead to a refinement of the model, enhancing the accuracy of the vehicle dynamics representation.

Modeling a non-developable 3D track that incorporates bumps and holes would represent a significant advancement in fully understanding the potential and limitations of these features. This approach would enhance the physical realism of the road-vehicle interaction, providing deeper insights into its dynamics.

Additionally, future developments could include an evaluation of design solutions for the various subsystems that control the additional degrees of freedom. These solutions must, however, take into account their practical feasibility for real-world systems, considering current technological limitations in sensors and actuators, as well as the challenges posed by their integration. 