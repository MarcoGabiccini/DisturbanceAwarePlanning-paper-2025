\section{Conclusions}
\label{sec:conclusions}

This work proposes and compares two robust planning strategies for minimum-time trajectory optimization in motorsport scenarios under uncertainty. The first approach, based on open-loop horizon-based covariance propagation, relies on a worst-case analysis of disturbance growth over a finite horizon to conservatively back off safety constraints. The second, a closed-loop covariance-aware method, incorporates a time-varying LQR feedback policy into the planning process, yielding a more realistic estimate of uncertainty evolution and tighter constraint enforcement.

Both schemes deliver reference trajectories suitable for human or artificial drivers. In autonomous applications the modelled controller can replicate the on-board implementation, providing highly accurate disturbance predictions; for human driving, fidelity improves with the extent to which the driver's behaviour can be approximated by the assumed time-varying LQR policy.

Quantitative results on a representative Barcelona-Catalunya sector highlight the trade-offs between robustness and performance. Open-loop planning, while conservative, ensures feasibility even in the absence of feedback, but incurs a higher performance cost due to larger safety margins. The closed-loop method achieves comparable safety guarantees with reduced conservativeness and improved lap-time performance, thanks to its ability to explicitly model the stabilizing action of the driver.
Finally, random simulations with noise realization, performed adopting the closed-loop approach, prove the effectiveness of the robustified constraints. A LQR-based controller tracking a robust trajectory planned with probabilistic constraint tightening exhibits significantly fewer track violations compared to one tracking the nominal optimal trajectory.

By embedding uncertainty propagation and feedback action directly in the planning stage, the proposed framework yields trajectories that are simultaneously fast and probabilistically safe, advancing minimum-time optimisation toward practical deployment in high-performance motorsport and autonomous racing. Future work will explore the generalization to more accurate vehicle models and experimental validation of the methodology. 