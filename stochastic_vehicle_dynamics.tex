\section{Stochastic vehicle dynamics}
\label{sec:stochastic_vehicle_dynamics}

We model the \emph{perturbed vehicle dynamics} as the following nonlinear continuous-time random dynamical system
\begin{align}
\dbx(t) = f(\bx(t), \bu(t)) + \bw(t),
\end{align}
where $\bu(t)$ are the deterministic control inputs, $\bw(t)$ is additive Gaussian white noise with zero mean and known covariance $\bQ(t)$, i.e. $\bw(t) \sim \calN(\bzero, \bQ(t))$, and the state $\bx(t)$ results in a Gaussian distribution with mean $\bmu(t)$ and covariance $\bP(t)$, so that $\bx(t) \sim \calN(\bmu(t), \bP(t))$.

Due to the symmetry of the probability density function (pdf) with respect to $\bmu(t)$, it is possible to represent the time evolution of the pdf as: i) the deterministic evolution of the mean $\bmu(t)$
\begin{align}\label{eq:meandynamics}
\dbmu(t) = f(\bmu(t), \bu(t)),
\end{align}
and ii) the time evolution of the state covariance matrix $\bP(t)$ along $\bmu(t)$, which can be expressed by the \emph{Lyapunov matrix differential equation}
\begin{align}\label{eq:dP}
\dbP(t) = \bA(t) \bP(t) + \bP(t) \bA^T(t) + \bQ(t),\quad \bP(0) = \bP_0 = \bP_0^T.
\end{align}
In~\eqref{eq:dP}, $\bP_0$ is the \emph{initial} state covariance and $\bA(t)$ is the usual shorthand notation for the Jacobian along the mean trajectory $\bmu(t)$, that is $\bA(t)=\bJ(\bmu(t),\bu(t))$, where $\bJ(\bx, \bu) = \frac{\pd f(\bx, \bu)}{\pd \bx}$.

As can be readily verified by differentiation~\cite{Gajic:LyapunovMatrixEquation:2010}, the analytical solution of~\eqref{eq:dP} has the form
\begin{align}\label{eq:P_STM}
\bP(t) = \bPhi(t,t_0) \bP_0 \bPhi(t,t_0)+\int_{t_0}^{t} \bPhi(t,\tau) \bQ(\tau) \bPhi^T(t,\tau) \dd \tau \quad \bP(0)=\bP_0,
\end{align}
where $\bPhi(t,t_0)$ is the \emph{state transition matrix}. This matrix encodes the evolution from $t_0$ to $t$ of a perturbation w.r.t. to a fiducial trajectory -- the mean $\bmu(t)$. In symbols, $\barbx(t) = \bPhi(t,t_0) \barbx(t_0)$, with $\barbx(\cdot) =\bx(\cdot) - \bmu(\cdot) $. In turn, the evolution of $\bPhi(t,t_k)$ from a generic $t_k$ to $t$ is driven by the following differential equation
\begin{align}\label{eq:STM}
\dbPhi(t,t_k) = \bA(t)\bPhi(t,t_k), \quad \bPhi(t_k, t_k) = \bI.
\end{align}
In geometric and orbital mechanics, see e.g.~\cite{Maruskin:DynamicalSystemsGeometric:2018} or~\cite{Tapley:StatisticalOrbitDetermination:2004}, the usual choice for statistical trajectory determination is to employ~\eqref{eq:meandynamics} along with~\eqref{eq:P_STM} and~\eqref{eq:STM}. In our case, since we use collocation integrators for stochastic trajectory planning, we follow a more direct approach -- motivated by~\cite{Gillis:PracticalMethodsApproximate:2015} -- by directly employing~\eqref{eq:meandynamics} and~\eqref{eq:dP}.

Accordingly, the continuous-time stochastic trajectory planning can be framed as the following nonlinear optimal control problem
\begin{align} 
& \underset{\bmu(t), \bu(t), \bP(t)}{\text{minimize}} \quad J(\bmu(t), \bu(t), \bP(t)) \label{eq:OCPcost} \\
& \text{s.t.} \quad \dbmu(t)           = f(\bmu(t), \bu(t)) \label{eq:OCPdyn} \\
& \phantom{\text{s.t.} \quad} \bmu(0)  = \bmu_0 \label{eq:OCPdynIC} \\
& \phantom{\text{s.t.} \quad} \dbP(t) = \bA(t) \bP(t) + \bP(t) \bA^T(t) + \bQ(t) \label{eq:OCPdP} \\
& \phantom{\text{s.t.} \quad} \bP(0)  = \bP_0 \succeq 0 \label{eq:OCPdPIC} \\ %& \phantom{\text{s.t.} \qquad} 0       \geq h_i(\bmu(t), \bu(t))
%+ \overbrace{\gamma \bigg[\underbrace{\na^T_{\bx} h_i \bP(t) \na_{\bx} h_i}_{\text{variance of $h_i(\bx)$}}\bigg]^{\frac{1}{2}}}^{\text{safety margin}},
&\phantom{\text{s.t.} \qquad} 0       \geq h_i(\bmu(t), \bu(t))
+ \be_i(\bmu(t), \bu(t), \bP(t)),
\quad i \in \calI \label{eq:OCPconstraints}
\end{align}
The cost function $J$ in~\eqref{eq:OCPcost} depends on the mean $\bmu(t)$, the controls $\bu(t)$ and the state covariance $\bP(t)$. The mean dynamics is expressed by~\eqref{eq:OCPdyn} with~\eqref{eq:OCPdynIC}, and the covariance dynamics is expressed by~\eqref{eq:OCPdP} with~\eqref{eq:OCPdPIC}. In eq.~\eqref{eq:OCPconstraints} the \emph{backoff terms} $\be_i$ account for the disturbances and serve the purpose of obtaining deterministic safety-margin types of constraint directly on $\bmu(t)$. $\calI$ is the set of indices defining the inequality constraints. This is equivalent to a linearization around the mean of the original \emph{chance constraint} on $\bx(t)$ expressed by $\Prob \{h_i(\bx) \leq 0\}\geq p$, where $p$ is the confidence level of constraint satisfaction. The backoff terms are given by
$\be_i = \gamma \sig_i$. The coefficient $\gamma = \Phi^{-1}(p)$ is the quantile function, where $\Phi(z)=\Prob \{Z\leq z\}$ is the cdf of a standard normal distribution $Z \sim \calN(0,1)$, and acts as a \emph{tuning knob}: the greater the confidence level $p$ required, the higher the gain $\ga$\footnote{For example, with $p=0.84$ $\ga = 1.0$, with $p=0.97$ $\ga = 2.0$, with $p=0.99$ $\ga = 3.0$}. The term $\sig_i = \big[\na^T_{\bx} h_i(\bmu) \bP(t) \na_{\bx} h_i(\bmu)\big]^{\frac{1}{2}}$ represents the standard deviation of the constraint $h_i(\bx)$ linearized around the mean, i.e. of random variable $h_i(\bmu)+\na^T_{\bx} h_i(\bmu) (\bx - \bmu)$, and follows immediately from the propagation rule of covariance.


%To compute the tangential components of the force exchanged by tyre and road, a formulation of the Pacejka's \emph{Magic Formula} ~\cite[Section~4.3.2]{Pacejka:book:2012} is used, which reads:
%\begin{equation}\label{eqn_mf}
%	[F_x,F_y]=\mf(\kappa,\alpha,F_z,\gamma).
%\end{equation}
%The Magic Formula $\mf$ computes the longitudinal and lateral forces $F_x,F_y$ as a function of the \emph{tyre slips} $\kappa,\alpha$~\cite[Section~1.2.1]{Pacejka:book:2012}. The formula is also sensitive to variations of vertical load $F_z$ and camber angle $\gamma$.
%
%The auxiliary frame $\sref{N}$ is introduced to compute quantities used in Eq.~\eqref{eqn_mf} and the (attempted) interpenetration $d$ (see Figure.~\ref{fig:tyre_wheel}) of the tyre in the ground, necessary to compute the dynamic vertical load $F_z$. The construction of the frame $\sref{N}$ is reported in Figure~\ref{fig:tyre_wheel}. It is worth noting that the frame $\sref{N}$ is fully described if $\sref{H}$ and $\sref{S}$ are known.
%The model used for the contact is a unilateral, penalty-based compliant tyre model, that considers the wheel as a radial spring with constant stiffness $k_t$ and constant damping coefficient $c_t$\footnote{The introduction of a more refined model with nonlinear spring and damping characteristics would not pose particular difficulties.}. The traditional role of compression in a spring is conducted by $d$, which is computed considering the road and the lowermost point of the non-deformed tyre. The resultant function that relate $d$, $\dot{d}$ and $F_z$ is:
%\begin{equation}\label{eqn_Fz}
%	F_z= F_0\log_2\left(1 + 2^{\frac{k_t d + c_t \dot{d}}{F_0}}\right)
%\end{equation}
%This function, with $F_0$ tuned to be negligible, is differentiable for $d=0$ and $\dot{d}=0$, enhancing numerical performance, and behaves as a spring-damper system, as requested, when the tyre is in contact with the ground. When the tyre detaches completely from the ground, the resultant vertical forces tends to zero. However, since its gradient is not zero, this formulation helps the optimization algorithm to figure out proper search directions. This allows to deal with jumps and other situations in which the tyre may lose contact with the ground.

\subsection{Discretization via direct collocation}
\label{sec:discretization}

The nonlinear optimal control problem~\eqref{eq:OCPcost} can be discretized by applying a suitable collocation integrator obtaining the following nonlinear program (NLP) 
\begin{alignat}{3}
\underset{\bmu_k,\bxi_k, \bu_k, \bP_k,\bz_k}{\text{minimize}} \,
& & & J_k(\bmu_k,\bxi_k, \bu_k) & & \label{eq:DOCPcost} \\
\hspace*{-2.0 cm}\text{s.t.} \quad
& \bzero      & = & \; \bPsimu_k(\bmu_{k-1},\bmu_k,\bxi_k, \bu_k,\bz_k),
& \quad & k = 1,\ldots, N \label{eq:DOCPdyn} \\
& \bmu_0      & = & \; \bar{\bmu}_0
& & \label{eq:DOCPdynIC} \\
& \bzero      & = & \; \bPsiP_k(\bmu_k,\bxi_k, \bu_k, \bP_{k-1},\bP_k,\bSi_k,\bz_k),
& \quad & k = 1,\ldots, N \label{eq:DOCPdP} \\
& \bP_0       & = & \; \bar{\bP}_0 \succeq 0
& & \label{eq:DOCPdPIC} \\
& \bzero      & = & \; \bOm_k(\bmu_k,\bxi_k, \bu_k,\bz_k),
& \quad & k = 0,\ldots, N \label{eq:DOCPpath} \\
& 0           & \geq & \; h_i(\bmu_k, \bu_k, \bz_k) + \be_i(\bmu_k, \bu_k, \bP_k, \bz_k),
& \quad & k = 1,\ldots, N;\; i \in \calI \label{eq:DOCPconstraints}
\end{alignat}
To perform the discretization, the track centerline is parameterized using a curvilinear abscissa $\alpha \in [0,1]$, and uniformly sampled at $N+1$ points $\alpha_0, \ldots, \alpha_N$.
Accordingly, $\bmu_k$ denotes the mean state at grid node $\alpha_k$, while $\bu_k$ and $\bz_k$ are assumed to be piecewise constant over each interval $[\al_k, \al_{k+1}]$.
These represent the controls and the algebraic variables, respectively. The $\bz_k$'s are introduced as direct handles for physically meaningful quantities such contact forces. Similarly, $\bP_k$ represents the covariance matrix at the $k$-th node.
Following the direct collocation approach, both the mean and covariance state trajectories are approximated, in the $k$-th interval, with polynomials $\pi_k(\tau)$, defined on the unit interval $\tau\in[0,1]$, and then scaled to match the width $h_k$ of the corresponding time step. On the unit interval, we select $d$ collocation points $\tau_1, \ldots, \tau_d$, associated with as many collocation states. Accordingly, we define $\bxi_k$ and $\bSi_k$ as the mean states and covariance matrices at the $d$ collocation points within each interval $[\al_k, \al_{k+1}]$, respectively. Therefore, $\bxi_k =
(\bxi_{k,1}, \ldots, \bxi_{k,d})$, and similarly $\bSi_k=(\bSi_{k,1}, \ldots, \bSi_{k,d})$. Equations~\eqref{eq:DOCPdyn} and~\eqref{eq:DOCPdP} represent the collocation and continuity equations for mean and covariance, respectively, with initial conditions represented by~\eqref{eq:DOCPdynIC} and~\eqref{eq:DOCPdPIC}. Eqs.~\eqref{eq:DOCPpath} are path equality constraints and~\eqref{eq:DOCPconstraints} are the robustified inequality constraints. 

In certain cases~\cite{Gillis:PracticalMethodsApproximate:2015}, positive-definiteness-preserving Lyapunov discretization schemes are used in~\eqref{eq:DOCPdP}. However, for sufficiently fine discretization, we found that integrating only the lower triangular part of $\bP(t)$ in~\eqref{eq:OCPdPIC}  (and reconstructing its strictly upper part accordingly), ensures that all $\bP_k$ remain symmetric and positive definite when the initial $\bP_0$ is symmetric and positive definite. In all our tests we used direct collocation with a cubic polynomial and Gauss-Legendre collocation points.
\subsection{External Wrenches}
\label{sec:extwren}
The external wrenches acting on each body, from wheels to chassis, are expressed in body-fixed components. Only gravity, aerodynamic forces, and road-tyre interactions are considered. Following~\cite{Guiggiani:book:2023}, the aerodynamic wrench in frame $\{B\}$ is $W_a=-\frac{1}{2}\rho S v_{gb,1}^2 [C_x, 0, C_z, 0, -h_0 C_x - a_f C_{zf} + a_r C_{zr}, 0]^T$, where $\rho$ is air density, $S$ is the frontal area, $v_{gb,1}$ the forward velocity (component of \emph{distal rigid-body twist} $V_{gb}$, see~\cite[p.~4]{Domenighini:Designs:2023}), and $C_x, C_z$ are drag/lift coefficients. $C_z$ is split into $C_{zf}=k_a C_z$ and $C_{zr}=(1-k_a)C_z$ by the aerodynamic balance coefficient $k_a$.

The total external wrench on a wheel is $W_{we}=\Ad_{g_{hn}}^*W_t+W_{w_h}$,
%\begin{equation}\label{eqn:Wwe}
%W_{we}=\Ad_{g_{hn}}^*W_t+W_{w_h},
%\end{equation}
where $W_t=[F_x, F_y, F_z, 0, 0, 0]^T$ is the ground wrench transferred from $\sref{N}$ to $\sref{H}$ by the \emph{starred} Adjoint operator, and $W_{w_h}=[F_{w_h}^T R_{hg}^T, 0, 0, 0]^T$ is the weight contribution with $F_{w_h}=[0, 0, -m_h g]^T$ in frame $\{G\}$, where $m_h$ is the wheel mass.

Knuckles experience no external forces, therefore:
\begin{equation}\label{eqn:Whe}
	W_{he}=[0,0,0,0,0,0]^T\quad\textrm{and}\quad W_{ke}=[0,0,0,0,0,0]^T.
\end{equation}
The chassis's external wrench is $W_{be}=W_a+W_{w_b}$, %\begin{equation}\label{eqn:Wbe}
%	W_{be}=W_a+W_{w_b},
%\end{equation}
where $W_a$ and $W_{w_b}$ are aerodynamic and weight contributions, respectively. For $W_{w_b}=[F_{w_b}^T R_{bg}^T, 0, 0, 0]^T$, the weight force is $F_{w_b}=[0, 0, -m_b g]^T$ in frame $\{G\}$, with $m_b$ as the chassis mass.
%The purpose of this section is to describe the external wrenches acting on each body, from the wheels to the chassis, each expressed in body-fixed components.
%Besides gravity, aerodynamic forces and the interactions between road and tyres are the only external forces considered.
%Adopting the notation in~\cite{Guiggiani:book:2023}, the aerodynamic contribution to the wrenches in frame $\{B\}$ is $W_a=-\frac{1}{2}\rho Sv_{gb,1}^2[C_x,0,C_z,0,-h_0C_x-a_fC_{zf}+a_rC_{zr},0]^T$, where $\rho$ is the air density, $S$ the frontal area of the vehicle, $v_{gb,1}$ the forward velocity, $h_0$ the nominal height of the CoM from ground and $a_f,a_r$ the distances of the CoM from the front and rear axle, respectively. The drag and lift coefficient $C_x$ and $C_z$ are dimensionless parameters that depend on the shape of the vehicle's body. $C_z$ can be further partitioned into $C_{zf}=k_a C_z$ and $C_{zr}=(1-k_a) C_z$ according to the \emph{aerodynamic balance coefficient} $k_a$, which is a parameter of the vehicle set-up.
%
%The total external wrench acting on a (general) wheel is:
%\begin{equation}\label{eqn:Wwe}
%	W_{we}=\Ad_{g_{hn}}^*W_t+W_{w_h},
%\end{equation}
%where $W_t=[F_x,F_y,F_z,0,0,0]^T$ are the ground forces computed in $\sref{N}$ and properly transferred to $\sref{H}$ components by the \emph{starred} Adjoint operator, while $W_{w_h} = [ F_{w_h}^T R^T_{hg}, 0,0,0 ]^T$, with $F_{w_h}=[0,0,-m_h g]^T$ in frame $\{G\}$, represents the weight contribution and $m_{h}$ is the wheel mass.
%
%Under our hypotheses, the knuckles are not subjected to external forces, thus \begin{equation}\label{eqn:Whe}
%	W_{he}=[0,0,0,0,0,0]^T\quad\textrm{and}\quad W_{ke}=[0,0,0,0,0,0]^T.
%\end{equation}
%
%The resultant external wrench acting on the chassis is therefore:
%\begin{equation}\label{eqn:Wbe}
%	W_{be}=W_a+W_{w_b},
%\end{equation}
%where $W_a$ and $W_{w_b}$ are, respectively, the contributions of aerodynamic and chassis weight forces, whose general expression is of the form $W_{w_b} = [ F_{w_b}^T R^T_{bg}, 0,0,0 ]^T$. Explicitly we have $F_{w_b}=[0,0,-m_b g]^T$ in frame $\{G\}$, where $m_b$ is the mass of the chassis.

\subsection{Recursive computation of the dynamics equations with ABA}
\label{sec:aba}
Thanks to the ABA algorithm, assuming known joint positions and velocities and joint forces and torques, it is possible to compute recursively and efficiently the accelerations of the joint variables. Following~\cite{Domenighini:Designs:2023}, twists are streamlined as follows: $V_{gb}^b = V_{gb} = V_b$, $V_{bh}^h=V_{bh}$, $V_{hw}^h=V_{hw}$. For the second model (DCC) the last expression is replaced with $V_{kw}^k=V_{kw}$ and $V_{hk}^k=V_{hk}$ is added. Given that the first model (without camber control) is a particular case of the DCC one, to avoid duplications only the algorithm for DCC is reported here.

The traditional ABA requires knowledge of the active forces and torques, denoted as $\tau_{ij}$, exerted by the parent body ($i$) on the child body ($j$). Its purpose is to solve the \emph{Forward Dynamics} problem of an articulated systems of rigid bodies. In our floating-base system, given the twist $V_b$ of the parent body $\{B\}$, the values and time derivatives of all joint variables ($q$ and $\dot{q}$), and the torque vector ($\tau$), the objective is to compute the joint accelerations ($\dot{V}_{b}$ and $\ddot{q}$).
%The fundamental concept is that of the inertia of an articulated body. This is represented by that of the root rigid body (original parent) plus the portion of the sub-trees, composed by the children bodies, whose inertia that can be structurally transferred backwards through the joints.
%This depends on the architecture, the pose and the joints' types.
The overall computation is performed in three sweeps defined recursively. One of the greatest computational benefits when it is employed to assemble symbolic dynamic equations, as in our case, is that \emph{no matrix inversion is required}, thus contributing to streamline the algebraic expressions.

The first step of the algorithm, as described in \textbf{Step 1}, performs the forward propagation of rigid-body velocities: from the root to the leaves of the tree.
%Twists are calculated as the sum of parent twist and the relative twist which is expressed consistently with the joint type. The interesting Jacobians are listed here: $J_{hk}=[0,0,0,1,0,0]^T$ and $J_{kw}=[0,0,0,0,1,0]^T$, and represent two revolute joints with axes $x$ and $y$ respectively.
In Algorithm~\ref{alg:step1} the terms ``Knuckle 1'' and ``Knuckle 2'' are used to identify the two halves of the knuckle, where, using kinematic trees terminology, the first is the parent of the second.
The Adjoint operator is defined as follows:
\begin{equation}\label{eq:Adjoint} \Ad_{g}=\left[\begin{array}{cc}R&\hat{d}R\\0&R\end{array}\right].
\end{equation}
In the case $g_{bh}\in SE(3)$ is considered, $R_{bh}\in SO(3)$ and $d_{bh}\in \bbR^3$ are functions of $(z,\delta)$. In the case of $g_{hk}$, the expressions are simply $d_{hk}=[0,0,0]^T$ and $R = \rotX(q_{hk})$, the latter being an elementary rotation about the common $x$-axis of the angle $q_{hk}$.

\begin{algorithm}[h]
	\floatname{algorithm}{Step}
	\caption{Forward Propagation of Velocity}\label{alg:step1}
	\begin{algorithmic}[1]
		\vspace{1mm}
		\For{$h=h_1,h_2,h_3,h_4$}
		\vspace{1mm}
		\State{$V_{bh}=J_{bh,z}\dot{z}+J_{bh,\delta}\dot{\delta}$}
		\vspace{1mm}
		\State{$V_h=\Ad_{g_{bh}}^{-1}V_b+V_{bh}$}\Comment{Knuckle 1 Rigid-Body Velocity}
		\vspace{1mm}
		\State{$V_{hk} = J_{hk}\dot{q}_{hk}$}
		\vspace{1mm}
		\State{$V_k=\Ad_{g_{hk}}^{-1}V_h+V_{hk}$}\Comment{Knuckle 2 Rigid-Body Velocity}
		\vspace{1mm}
		\State{$V_{kw}=J_{kw}\omega$}
		\vspace{1mm}
		\State{$V_w=V_k+V_{kw}$}\Comment{Rim Rigid-Body Velocity}
		\vspace{1mm}
		\EndFor
		\vspace{1mm}
	\end{algorithmic}
\end{algorithm}

The second step of the algorithm, as detailed in \textbf{Step 2}, performs the backward propagation of inertia and bias terms.
\begin{algorithm}[h]
	\floatname{algorithm}{Step}
	\caption{Backward Propagation of Articulated Inertia and Bias}\label{alg:step2}
	\begin{algorithmic}[1]
		\vspace{1mm}
		\For{$h=h_1,h_2,h_3,h_4$}
		\vspace{1mm}
		\State{$\hat{M}_w=M_w$}\Comment{Rim Articulated Inertia}
		\vspace{1mm}
		\State{$\hat{B}_w=-W_{we}+\ad_{V_w}^*M_wV_w$}\Comment{Rim Articulated Bias}
		\vspace{1mm}
		\State{$\bar{M}_w=\hat{M}_w-\dfrac{\hat{M}_wJ_{hw}J_{hw}^T\hat{M}_w}{J_{hw}^T\hat{M}_wJ_{hw}}$}
		\vspace{1mm}
		\State{$\bar{B}_w=\hat{B}_w-\dfrac{\hat{M}_wJ_{hw}J_{hw}^T\hat{B}_w}{J_{hw}^T\hat{M}_wJ_{hw}}-\bar{M}_w\ad_{V_{hw}}V_w$}
		\vspace{1mm}
		\State{$\hat{M}_k= \bar{M}_w$
		}\Comment{Knuckle 2 Articulated Inertia}
		\vspace{1mm}
		\State{$\hat{B}_k= \bar{B}_w$
		}\Comment{Knuckle 2 Articulated Bias}
		\vspace{1mm}
		\State{$\hat{M}_h= \Ad_{g_{hk}}^*\hat{M}_k\Ad_{g_{hk}}^{-1}$
		}\Comment{Knuckle 1 Articulated Inertia}
		\vspace{1mm}
		\State{$\hat{B}_h= \Ad_{g_{hk}}^*\bigl(\hat{M}_k(J_{hk}\ddot{q}_{hk}-\ad_{V_{hk}}\Ad_{g_{kh}}V_k)+\hat{B}_k\bigr)$
		}\Comment{Knuckle 1 Articulated Bias}
		\vspace{1mm}
		\State{$\bar{M}_h=\hat{M}_h-\dfrac{\hat{M}_hJ_{bh,z}J_{bh,z}^T\hat{M}_h}{J_{bh,z}^T\hat{M}_hJ_{bh,z}}$}
		\vspace{1mm}
		\State{$\bar{B}_h=\hat{B}_h-\dfrac{\hat{M}_hJ_{bh,z}(J_{bh,z}^T\hat{B}_h-\tau)}{J_{bh,z}^T\hat{M}_hJ_{bh,z}}-\bar{M}_h(\ad_{V_{bh}}V_h-\dot{J}_{bh,z}\dot{z}-\dot{J}_{bh,\delta}\dot{\delta}-J_{bh,\delta}\ddot{\delta})$}
		\vspace{1mm}
		\EndFor
		\vspace{1mm}
		\State{$\hat{M}_b=M_b+\displaystyle\sum_h\Ad_{g_{bh}}^*\bar{M}_h\Ad_{g_{bh}}^{-1}$}\Comment{Chassis Articulated Inertia}
		\vspace{1mm}
		\State{$\hat{B}_b=-W_{be}+\ad_{V_b}^*M_bV_b+\displaystyle\sum_h\Ad_{g_{bh}}^*\bar{B}_h$}\Comment{Chassis Articulated Bias}
		\vspace{1mm}
	\end{algorithmic}
\end{algorithm}

%Both inertia and bias are expressed in the body-fixed components.
In the case the DCC architecture is considered, the camber angle acceleration $\ddot{q}_{hk}$ is known, since $q_{hk(t)}$ is the assumed input, while the torque $\tau_{hk}$ has to be computed together with the rest of unknown accelerations, which are the outcome of the third sweep. This leads to a \emph{Hybrid Dynamics} problem that requires to carefully revisit the ABA's second and third steps.
The approach used is inspired by~\cite[Section~2.3]{Kim:notes:2012} where the the notation and conventions here employed were adapted according to~\cite{Domenighini:Designs:2023}.
%Specifically, apart from the labels assigned to the quantities, the primary difference between the two methods is the ordering of the twist vector. In this work, the twist vector is defined as $V = [v,\omega]^T$, whereas in the other method, it is defined as $V=[\omega,v]^T$. This difference might seem minor but leads to significant changes in the definitions of Adjoint operators, inertia matrices, bias and wrench vectors.

%For a generic articulated body the Newton-Euler equations can be expressed in the form:
%\begin{equation}\label{eqn:NE}
%	W=\hat{M}\dot{V}+\hat{B}.
%\end{equation}
%For a leaf node of the kinematic tree, having no children, the articulated inertia $\hat{M}$ is simply the generalized inertia $M$ of the body itself. Consistently, the bias $\hat{B}=B$, where:
%\begin{equation}
%	B=-W_e+\ad_V^*MV,
%\end{equation}
%and where $W_e$ is the resultant external wrench applied to the body and $\ad_V^*MV$ accounts for generalized gyroscopic and centrifugal forces, which are bi-linear in $V$. According to~\cite[Section~4.3.3]{Murray:book:1994} $\ad^*=-\ad^T$ has been defined. (If $v$ and $\omega$ are the translational and rotational component of $V$, then $\ad^*_V=\left[\begin{smallmatrix}\hat{\omega}&0\\\hat{v}&\hat{\omega}\end{smallmatrix}\right]$).

Starting from the leaf nodes, which in our case are represented by the wheels ($\sref{W}$), after providing information such as external wrenches, twists, joint geometry, body masses, joint forces / torques for traditional connections, and time evolution of variables defining kinematically imposed joints, the backward propagation of inertia and bias can be performed.

Finally the last step, denoted as \textbf{Step 3}, starting from the root node, computes joint variable accelerations and consequently body twists derivatives. For the kinematically controlled camber joints, the torques $\tau_{hk}$ needed to maintain the desired state of motion (DCC architecture) are computed as a by-product of the local inverse dynamics. This information is crucial to prevent actuators from  saturating or operating outside their bandwidth in the planned trajectories.

The root twist acceleration is determined by solving equation $W=\hat{M}\dot{V}+\hat{B}$, considering that the chassis is free-floating. The other unknown quantities, such as variable accelerations and torques, are computed projecting the Newton-Euler equations of each body along the respective Jacobian and then solving for the variable. See also~\cite{Domenighini:Designs:2023} for more details in the derivation of the equations.

\begin{algorithm}[h]
	\floatname{algorithm}{Step}
	\caption{Forward Propagation of Acceleration}\label{alg:step3}
	\begin{algorithmic}[1]
		\State{$\dot{V}_b=-\hat{M}_b^{-1}\hat{B}_b$}\Comment{Chassis Rigid-Body Acceleration}
		\vspace{1mm}
		\For{$h=h_1,h_2,h_3,h_4$}
		\vspace{1mm}
		\State{$\ddot{z}=-\dfrac{J_{bh,z}^T\bigl(\hat{B}_h+\hat{M}_h(\Ad_{g_{bh}}^{-1}\dot{V}_b-\ad_{V_{bh}}V_h+\dot{J}_{bh,z}\dot{z}+\dot{J}_{bh,\delta}\dot{\delta}+J_{bh,\delta}\ddot{\delta})\bigr)-\tau}{J_{bh,z}^T\hat{M}_hJ_{bh,z}}$}
		\vspace{1mm}
		\State{$\dot{V}_{bh}=\dot{J}_{bh,z}\dot{z}+\dot{J}_{bh,\delta}\dot{\delta}+J_{bh,z}\ddot{z}+J_{bh,\delta}\ddot{\delta}$}
		\vspace{1mm}
		\State{$\dot{V}_h=\Ad_{g_{bh}}^{-1}\dot{V}_b-\ad_{V_{bh}}V_h+\dot{V}_{bh}$}\Comment{Knuckle 1 Rigid-Body Acceleration}
		\vspace{1mm}
		\State{$\tau_{hk} = J_{hk}^T\bigl(\hat{B}_k+\hat{M}_k(\Ad_{g_{kh}}\dot{V}_h-\ad_{V_{hk}}V_k + J_{hk}\ddot{q}_{hk})\bigr)$}
		\vspace{1mm}
		\State{$\dot{V}_{bh} = J_{hk}\ddot{q}_{hk}$}
		\vspace{1mm}
		\State{$\dot{V}_k=\Ad_{g_{hk}}^{-1}\dot{V}_h-\ad_{V_{hk}}V_k+\dot{V}_{bh}$}\Comment{Knuckle 2 Rigid-Body Acceleration}
		\State{$\dot{\omega}=-\dfrac{J_{kw}^T\bigl(\hat{B}_w+\hat{M}_w(\dot{V}_k-\ad_{V_{kw}}V_w)\bigr)}{J_{kw}^T\hat{M}_wJ_{kw}}$}
		\vspace{1mm}
		\EndFor
		\vspace{1mm}
	\end{algorithmic}
\end{algorithm}

\subsection{Dynamic Model}
The dynamics equations, assembled via the procedure described  in Sec.~\ref{sec:aba}, are implemented in the software code by using a state-space representation $\dot{x}=f(x,u)$.
%\begin{equation}\label{eqn:f}
%	\dot{x}=f(x,u).
%\end{equation}
The state vector is $x = (q_{gb};V_{gb};z;\dot{z};\omega;q_{hk};\dot{q}_{hk})\in\bbR^{32}$, with components defined in previous sections. As already mentioned, in the model the camber joint is kinematically controlled as per the DCC architecture. Hence, the associated control vector is $u=(T_a;T_b;\delta;\ddot{q}_{hk})\in\bbR^{16}$, where each element is in $\bbR^{4}$ and collects the controls of the branch to each wheel. The torque is split into acceleration and braking components, allowing for minor modifications to model on-board motors coupled with traditional brakes and/or in-wheel motors as well. These components have their reaction torques in two different bodies in case of on-board motors and traditional brakes: the chassis and the knuckles, respectively. The $\delta$ vector collects the four translations of the steering rod attachment points, as described in Section~\ref{sec:suspanal}, while for the vector $\ddot{q}_{hk}$ no further explanation should be necessary. Clearly, this representation is valid for the model with active camber. The simpler model (no DCC) shares with the former the first 24 components of the state vector and the first 12 of the control vector.

The time derivative of the state vector has also been expressed previously except for the first element, $q_{gb}$. Its derivative is a function of the state and can be recovered using a standard procedure~\cite{Domenighini:Designs:2023} as follows:
\begin{equation}\label{eqn:dotqgb}
	\dot{q}_{gb}=J_{gb}^{-1}(q_{gb})V_{gb},
\end{equation}
where $J_{gb}(q_{gb})\in\bbR^{6\times6}$ is the Jacobian of the virtual joint connecting the ground to the chassis, and it is computed via the PoE parameterization of the transformation $g_{gb}$.
%, similarly to $J_{bh}$ in Eq.~\eqref{eqn:Jbh}. 